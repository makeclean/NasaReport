
%%%%%%%%%%%%%%%%%%%%%%%%%%%%%%%%%% Introduction %%%%%%%%%%%%%%%%%%%%%%%%%%%%%%%%
\section{Introduction}
This report covers the work performed at UW in Wyle Contracts xx \& xx. The work
can be broken down into the following sections;

\begin{itemize}
  \item{CAD Interface Development}
  \item{Workflows \& Tools}
  \item{Benchmarking Activities}
  \item{Analysis}
\end{itemize}

Section \ref{sec:cad_interfaces} discusses the development work that was
performed the development of The University of Wisconsin Unified Workflow
(\ref{sec:uwuw}) a workflow that facilities the use of DAGMC geometries,
material definitions \& tally specifications in multiple Monte Carlo (MC) codes.
The development of FluDAG is covered in Section \ref{sec:fludag}. Updates made
to the DagSolid implementation of a Tessellated Solid for Geant4 can be found
in Section \ref{sec:dagsolid}, related to DagSolid the DagGeant4 executable
can be found in Section \ref{sec:daggeant4}.
\\
\\
Section \ref{sec:workflows} discusses the workflows \& tools that were developed
as part of this work. This includes the develpoment of the \texttt{ReadOBJ}
method in MOAB for the reading of OBJ files, found in Section \ref{sec:readobj}.
A tool to generate hierarchical information from OBJ like geometries was
developed called \texttt{GenerateHierarchy} and is discussed in Section
\ref{sec:genhi}. UW made contributions to GCR source development in the
SRAGCodes repository and is discussed in Section \ref{sec:sragcodes}.
\\
\\
In order to prove the correctness of FluDAG several benchmarks were performed
and are discussed in Section \ref{sec:benchmarking}.
\\
\\
UW also provided results for a number of different sets of analysis, these can
be found in Section \ref{sec:analysis}.
